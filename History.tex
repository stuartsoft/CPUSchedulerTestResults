\section{Brief History}

In 20+ years of Linux development, the Kernel scheduler has undergone several major revisions. Among these revisions are three primary scheduling systems.

\textbf{O(N) scheduler} This scheduling system ran through every task in the run queue during each scheduling event, picking and choosing which processes to run. Despite the simplicity of rating and picking processes sequentially at each scheduling routine, this method was not suitable for a wide variety of hardware and not ready for multi-core systems. [1]

Linux 2.6.x Saw the change of several different scheduling systems as it served the Linux community from 2003 until 2016 through it's long term support variant. Kernel iterations up to 2.6.22 included the \textbf{O(1) scheduler} by Ingo Molnar, an effectively constant time scheduling algorithm designed to replace to formerly costly O(N) scheduler. O(1) made use of a scheduling priority run queue, which allowed it to simply pick the top-most process in constant time. As the Kernel matured however, the heuristics of the O(1) run queue became difficult to manage and beginning in 2.6.23 the \textbf{Completely Fair Scheduler} (CFS) by Ingo Molnar was introduced. [1]

With CFS came a large change in process ordering and organization. CFS operates through a modified variant of round robin scheduling and utilizes a self-balancing red-black tree system. All operations in a red-black tree can be performed in O(log n), including adding nodes, deleting nodes and finding nodes. The CFS red-black tree organizes processes that need the cpu the most on the left of the tree and process that need the cpu the least on the right. A process's need for the cpu is inversely proportional to it's "virtual runtime". The virtual runtime for a given process is calculated as the amount of time assigned to a given process. [1]

Additionally, CFS provides a method of dynamic process prioritization through the use of "nice" values. This allows the system a secondary way to adjust a process's need for the cpu and allows users to arbitrarily give certain processes higher priority over others. CFS also has support for various other methods of load balancing between cores, sharing runtime quotas between runqueues and various other rules and features.

Other schedulers such as Con Kolivas' \textbf{Rotating Staircase Deadline Scheduler} and his \textbf{BFS scheduler} make use of alternative methods for organizing and scheduling processes in the Kernel. These are frequently offered in the form of a Kernel patch or sometimes as a completely packaged Kernel, where the Kernel variant is represented by the author's initials, such as \textbf{linux-ck}. [1]