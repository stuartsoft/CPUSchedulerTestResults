\section{Introduction}
	Nearly any modern computing device today has a well balanced and finely tuned scheduling system. As a result, operating system kernels must be widely configurable and flexible among a large array of hardware configurations and applications. Additionally, while it may be easy to make a scheduler that performs with very high throughput, it is a much larger task to build a scheduling system that caters to all types of scheduling criteria (turnaround time, responsiveness, wait time) without causing process starvation or instabilities in system operation.
	
	This paper presents an overview of a few of the main components of cpu scheduling under the Linux kernel as well as a set of tests and benchmarks detailing the observed performance of various scheduler modifications. Additionally, it aims to provide an overview of the Linux CFS scheduling operation as well as discuss several other alternative real-world scheduling algorithms.